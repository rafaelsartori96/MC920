% Rafael Sartori M. dos Santos, 186154
\documentclass[brazilian,a4paper,twocolumn]{article}

% Título
\title{MC920 -- Trabalho 2}
\author{Rafael Sartori M. Santos, 186154}
\date{1 de outubro de 2019}

% Configuração do documento
\setlength{\parskip}{3pt}
\usepackage[utf8]{inputenc} % tipo de documento UTF-8
\usepackage{mathtools} % permitir expressões matemáticas
\usepackage{breqn} % equações quebradas em várias linhas automaticamente
\usepackage{babel} % configuração da lingua portuguesa
\usepackage{caption} % para legenda de tabelas e figuras
\usepackage[
    pdfauthor={Rafael Sartori M. Santos},
    pdftitle={Trabalho 2 -- MC920},
    pdfproducer={LaTeX (texlive) com hyperref}
]{hyperref} % para links externos (href)
\usepackage{cleveref} % para referenciar tabelas e figuras melhor
\usepackage{indentfirst} % indentação de todo primeiro parágrafo
\usepackage{graphicx} % para adicionar imagens
\usepackage{subcaption} % para imagens ficarem lado a lado
% Usamos geometry pois dá mais espaço que fullpage
%\usepackage{geometry} % alterar geometria do papel
%\geometry{a4paper,left=1.7cm,right=1.7cm,top=1cm,bottom=2.0cm} % menor margem
\usepackage{fullpage} % utilizamos uma versão com menos espaçamento nas bordas

% Início do documento
\begin{document}

\maketitle


\section{Introdução}

    Neste trabalho, temos que avaliar e comparar diferentes métodos de limiarização (locais e globais). Aplicarei cada transformação em imagens monocromáticas em formato PGM fornecidas pelo professor H. Pedrini como sugere o enunciado. O resultado analisarei quanto aos contornos dos objetos, detalhes mantidos da imagem e ruído.

    Executarei esse processamento utilizando Python com as bibliotecas padrão, \href{https://opencv.org/}{\emph{OpenCV}}, \href{https://matplotlib.org/}{\emph{Matplotlib}} e \href{https://numpy.org/}{\emph{NumPy}}.


\section{Método}

    As imagens fornecidas pelo professor foram armazenadas na pasta de entrada \texttt{imgs/} sem que fosse necessária qualquer conversão.

    Para realizar o processamento digital, as bibliotecas de Python que utilizei foram:
    \begin{itemize}
        \item \emph{OpenCV} para abrir e salvar imagens;
        \item \emph{NumPy} para aplicar transformações à imagem;
        \item \emph{Matplotlib} para produzir histogramas;
        \item Alguns módulos da padrão para interpretação da entrada (configuração de parâmetros para filtros, determinar imagem de saída, produzir informações como histogramas e relação entre pretos e brancos).
    \end{itemize}

    O código que interpreta as entradas e chama a função que corresponde ao filtro está em \texttt{main.py}. Algumas funções genéricas (aplicação de filtro, abrir e salvar imagens) estão em \texttt{util.py}. Os outros arquivos correspondem cada um a um filtro diferente de limiarização.

    Como cada filtro possui diferente número de parâmetros, utilizei o recurso de argumentos variáveis de Python (o dicionário \texttt{**kwargs}) para que consiga de forma fácil, genérica e sem depender da ordem produzir os parâmetros dos filtros com valores padrões quando não mencionados.


\section{Métodos de limiarização}

    A limiarização ocorre através da comparação do ponto em que estamos considerando, $(x, y)$, com um limiar $T(x, y)$. De tal forma a produzir a imagem $g$ binária a partir de $f$ seguindo a \cref{eq:limiarizacao}.

    \begin{equation}
    \label{eq:limiarizacao}
        g(x, y) =
        \begin{cases}
            1       & \text{se $f(x, y) \geq T(x, y)$} \\
            0       & \text{caso contrário}
        \end{cases}
    \end{equation}

    Quando a limiarização é local, podemos especificar o tamanho da vizinhança quadrada $Z_{n,n}$ alterando o valor de $n$ (\texttt{--dimensao} no programa), que deve ser ímpar para que o filtro seja aplicável de forma igual numa imagem discreta.

    \subsection{Global}

        A limiarização global é feita através de um limiar a ser aplicado em toda imagem, representado pela \cref{eq:global}.

        \begin{equation}
        \label{eq:global}
            T(x, y) = k
        \end{equation}

        O parâmetro $k$ é usado no programa para determinar o limiar global, padrozinado em $128$.

    \subsection{Bernsen}

        A limiarização local de Bernsen utiliza o máximo e mínimo dos pontos de $Z$ de acordo com a \cref{eq:bernsen}. Não possui parâmetros.

        \begin{equation}
        \label{eq:bernsen}
            T(x, y) = (Z_{min} + Z_{max}) / 2
        \end{equation}

    \subsection{Niblack}
    \label{sec:niblack}

        A limiarização local de Niblack utiliza a média $Z_{avg}$ e o desvio padrão $Z_{std}$ dos pontos de $Z$ de acordo com a \cref{eq:niblack}. Possui um parâmetro $k$ para o peso dado ao desvio padrão, padronizado em $0.8$ como no enunciado.

        \begin{equation}
        \label{eq:niblack}
            T(x, y) = Z_{avg} + k \cdot Z_{std}
        \end{equation}

    \subsection{Sauvola e Pietikäinen}
    \label{sec:sauvola-pietikainen}

        A limiarização local de Sauvola-Pietikäinen utiliza a média $Z_{avg}$ e o desvio padrão $Z_{std}$, como em \ref{sec:niblack}, através da \cref{eq:sauvola-pietikainen}, com a intenção de produzir melhores resultados sob má iluminação.

        \begin{equation}
        \label{eq:sauvola-pietikainen}
            T(x, y) = Z_{avg} \cdot \left[ 1 +  k \cdot \left( \frac{Z_{std}}{R} - 1\right) \right]
        \end{equation}

        Possui dois parâmetros: $k$ e $R$, padronizados respectivamente em $0.5$ e $128$.

    \subsection{Phansalskar, More e Sabale}

        É uma variação de \ref{sec:sauvola-pietikainen}, porém pretende-se lidar melhor com imagens de baixo contraste. Então utiliza também a média e desvio padrão, como vemos na \cref{eq:pms}.

        \begin{multline}
        \label{eq:pms}
            T(x, y) = Z_{avg} \cdot \Biggl[ 1 +  p \cdot \exp{ \left( -q \cdot Z_{avg} \right) } \Biggr. \\ \Biggl. + k \cdot \left( \frac{Z_{std}}{R} - 1 \right) \Biggr]
        \end{multline}

        No código, é abreviado por \texttt{PMS} e utiliza todos os possíveis 4 parâmetros: $k$, $R$, $p$ e $q$, padronizados respectivamente por $0.25$, $0.5$, $2$ e $10$.

    \subsection{Contraste}

        O método local do constrate utiliza a distância relativa ao ponto mais escuro e mais claro da vizinhança: se está mais próximo de um ponto claro, é claro; se está de um ponto escuro, é escuro.

        \begin{equation}
        \label{eq:contraste}
            F =
            \begin{cases}
                0       & \text{se $\mathopen|I - Z_{max}\mathclose| \geq \mathopen|I - Z_{min}\mathclose|$} \\
                1       & \text{caso contrário}
            \end{cases}
        \end{equation}

        O ponto final $F = g(x, y)$ da imagem pode ser expresso pela \cref{eq:contraste} onde $I$ é o ponto inicial da imagem $f(x, y)$. Não requer qualquer parâmetro.

    \subsection{Média e mediana}

        Consideramos a média ou a mediana dos pontos da vizinhança. Sendo esse valor o limiar $T(x, y)$, basta comparamos com o ponto em que estamos $f(x, y)$ como em \cref{eq:limiarizacao}. Também não requer qualquer parâmetro.


\section{Resultados obtidos e análise}

    A avaliação dos resultados mostrou-se desafiadora: com o bem diverso conjunto de imagens de entrada, é possível notar após análise superficial que produzir um resultado satisfatório com apenas um método ``faz-tudo'' em todos os casos não foi possível. Podemos, no entanto, destacar em que partes cada método desempenha melhor e tentar explicar os motivos para isso.

    A abordagem que tomei, então, foi por agrupamento de imagens de entrada (ao invés de agrupar por método), sintetizando ao final um resumo do que os métodos foram capazes. Nesse rumo, poderemos ter uma visão geral das condições e implicar as características do método.


\section{Conclusão}

    a

\end{document}
