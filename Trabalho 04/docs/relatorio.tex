% Rafael Sartori M. dos Santos, 186154
\documentclass[brazilian,a4paper,twocolumn]{article}

% Título
\title{MC920 -- Trabalho 4}
\author{Rafael Sartori M. Santos, 186154}
\date{31 de outubro de 2019}

% Configuração do documento
\setlength{\parskip}{3pt}
\usepackage[utf8]{inputenc} % tipo de documento UTF-8
\usepackage{mathtools} % permitir expressões matemáticas
\usepackage{breqn} % equações quebradas em várias linhas automaticamente
\usepackage{babel} % configuração da lingua portuguesa
\usepackage{caption} % para legenda de tabelas e figuras
\usepackage[
    pdfauthor={Rafael Sartori M. Santos},
    pdftitle={Trabalho 4 -- MC920},
    pdfproducer={LaTeX (texlive) com hyperref},
    hidelinks
]{hyperref} % para links externos (href)
\usepackage{cleveref} % para referenciar tabelas e figuras melhor
\usepackage{indentfirst} % indentação de todo primeiro parágrafo
\usepackage{graphicx} % para adicionar imagens
\graphicspath{{../imgs/}} % atalho para o caminho das imagens
\usepackage{float} % para fixar posição de imagens
\usepackage{subcaption} % para imagens ficarem lado a lado
% Usamos geometry pois dá mais espaço que fullpage
%\usepackage{geometry} % alterar geometria do papel
%\geometry{a4paper,left=1.7cm,right=1.7cm,top=1cm,bottom=2.0cm} % menor margem
\usepackage{fullpage} % utilizamos uma versão com menos espaçamento nas bordas
\usepackage{verbatim} % pacote para incluir arquivos em verbatim
\usepackage{mdframed} % para enquadrar coisas
\usepackage[bitstream-charter]{mathdesign} % Mudamos a fonte para Charter BT
\usepackage[T1]{fontenc} % Mudamos a fonte para Charter BT

% Início do documento
\begin{document}

\maketitle


\section{Introdução}

    O objetivo do trabalho é a codificação e decodificação de mensagens dentro de imagens e verificar se as novas informações produzem algum artefato visual na imagem. Como exigido pelo enunciado, haverá 2 programas: um para codificação e outro para decodificação.

    Faremos esses programas utilizando Python com as bibliotecas \href{https://opencv.org/}{\emph{OpenCV}}, \href{https://numpy.org/}{\emph{NumPy}}, \href{https://pycrypto.org/}{\emph{pycrypto}} e padrão.

    Também utilizaremos \href{https://www.gimp.org/}{\emph{GIMP}} para alterar o formato da imagem que utilizaremos.


\section{Metodologia}

    No programa de codificação, precisamos receber uma imagem e um arquivo a ser codificado. Então abrimos essa imagem, alteramos um único \textit{bit} para cada ponto e camada da imagem, escrevendo a mensagem que desejamos, e salvamos a imagem de saída.

    No de decodificação, recebemos a imagem com a mensagem codificada, abrimos e isolamos o \textit{bit} desejado, acumulando num vetor binário que será posteriormente convertido a texto para salvarmos.

    Com esses passos gerais, começamos preparando a imagem, alterando seu formato em um programa externo.

    \subsection{Preparação da imagem}
        É um requisito trabalhar de forma \textit{lossless} (sem perdas) para manter todas as informações da imagem intactas, garantindo a integridade da mensagem. Isto é, algoritmos de compressão \textit{lossy} podem degradar a imagem e possivelmente causar a corrupção da mensagem.

\end{document}
