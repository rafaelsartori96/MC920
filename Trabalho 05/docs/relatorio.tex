% Rafael Sartori M. dos Santos, 186154
\documentclass[brazilian,a4paper,twocolumn]{article}

% Título
\title{MC920 -- Trabalho 5}
\author{Rafael Sartori M. Santos, 186154}
\date{21 de novembro de 2019}

% Configuração do documento
\setlength{\parskip}{3pt}
\usepackage[utf8]{inputenc} % tipo de documento UTF-8
\usepackage{mathtools} % permitir expressões matemáticas
\usepackage{breqn} % equações quebradas em várias linhas automaticamente
\usepackage{babel} % configuração da lingua portuguesa
\usepackage{caption} % para legenda de tabelas e figuras
\usepackage[
    pdfauthor={Rafael Sartori M. Santos},
    pdftitle={Trabalho 5 -- MC920},
    pdfproducer={LaTeX (texlive) com hyperref},
    hidelinks
]{hyperref} % para links externos (href)
\usepackage{cleveref} % para referenciar tabelas e figuras melhor
\usepackage{indentfirst} % indentação de todo primeiro parágrafo
\usepackage{graphicx} % para adicionar imagens
\graphicspath{{../imgs_in/}{../imgs_out/}} % atalho para o caminho das imagens
\usepackage{float} % para fixar posição de imagens
\usepackage{subcaption} % para imagens ficarem lado a lado
% Usamos geometry pois dá mais espaço que fullpage
%\usepackage{geometry} % alterar geometria do papel
%\geometry{a4paper,left=1.7cm,right=1.7cm,top=1cm,bottom=2.0cm} % menor margem
\usepackage{fullpage} % utilizamos uma versão com menos espaçamento nas bordas
\usepackage{verbatim} % pacote para incluir arquivos em verbatim
\usepackage{mdframed} % para enquadrar coisas
\usepackage[bitstream-charter]{mathdesign} % Mudamos a fonte para Charter BT
\usepackage[T1]{fontenc} % Mudamos a fonte para Charter BT

% Início do documento
\begin{document}

\maketitle


\section{Introdução}

    O objetivo do trabalho é avaliar o procedimento PCA (\textit{Principal Component Analysis}) como efetivo método de compressão de imagens através da redução de componentes menos significativos.

    Faremos isso através de um programa Python que utiliza as bibliotecas \href{https://opencv.org/}{\emph{OpenCV}}, \href{https://numpy.org/}{\emph{NumPy}} e padrão. Utilizamos também o \href{https://www.gimp.org/}{\emph{GIMP}} como \textit{software} para manipular algumas imagens antes da execução do programa.


\section{Metodologia}

    Analisaremos a compressão feita com a redução de componentes dados pelo PCA nas imagens dadas pelo professor no enunciado do trabalho e em uma imagem de grande dimensão (comparada às outras) capturada por um celular comum moderno.

    Todas as imagens utilizadas estavam no formato \texttt{PNG}, como especificado no enunciado, para não serem afetadas pela compressão com perdas de outros métodos. Durante o desenvolvimento do projeto, sobreescrevemos as imagens originais utilizando a biblioteca \emph{OpenCV} (a mesma que utilizaremos para salvar as imagens alteradas), de forma a minimizar os efeitos das diferentes implementações do formato \texttt{PNG}.

    Começamos o programa descrevendo diversas opções de entrada para o usuário: qual imagem será utilizada, o parâmetro $k$ para redução de componentes, onde salvaremos a imagem final, o relatório (em texto) de análise da compressão e ainda se utilizaremos a implementação para SVD (\textit{Singular Value Decomposition}, uma das maneiras de se fazer PCA) já feita de \emph{NumPy} ou, caso contrário, se utilizaremos a nossa implementação.

    Utilizando essas opções, abrimos a imagem através de \emph{OpenCV}, isolando as camadas de cores para serem tratadas individualmente e as transformando em vetor de pontos flutuantes.

    \subsection{Utilizando SVD de \emph{NumPy}}

        Na implementação utilizando SVD de \emph{NumPy}, fazemos a decomposição e, posteriormente, pegamos apenas os $k$ primeiros componentes mais significativos para refazer a imagem. Isso pode ser visto na \cref{eq:pca-svd}, considerando $g$ a imagem final e $f$, a inicial.

        \begin{subequations}
            \label{eq:pca-svd}
            \begin{equation}
                \label{eq:pca-svd-decomposicao}
                U, D, V^T = \texttt{svd}(f)
            \end{equation}
            \begin{equation}
                \label{eq:pca-svd-recombinacao}
                g = U_{[:, 0:k]} D_{[0:k]} V^T_{[0:k, :]}
            \end{equation}
        \end{subequations}

        Na \cref{eq:pca-svd-decomposicao}, \emph{NumPy} já ordena decrescentemente o resultado de SVD, o que é mais amigável as aplicações que utilizam isso para fazerem PCA. Desta forma, é bem rápido e fácil de implementar um algoritmo de compressão que utiliza PCA, mas não vemos com detalhes as operações realizadas nem a interpretação não usual da imagem como componentes.

    \subsection{Implementação própria}

        Já na implementação própria de PCA, ao contrário, fazemos a compressão em detalhes, podemos entender melhor a análise interpretando a matriz da imagem como um vetor de informações, onde cada linha é um componente e cada coluna diferentes atributos desse componente.

        Com a imagem inicial $X$, na \cref{eq:pca-1}, removemos de cada componente a média dos atributos (representada na matriz $\overline{X_j}$), de forma a obtermos uma média zero individualmente. Com isso, podemos obter a matriz de covariância $X_n$, como em \cref{eq:pca-covariancia}.

        \begin{subequations}
            \label{eq:pca-1}
            \begin{equation}
                \label{eq:pca-media}
                X_n = X - \overline{X_j}
            \end{equation}
            \begin{equation}
                \label{eq:pca-covariancia}
                \Sigma = X_n X_n^T
            \end{equation}
        \end{subequations}

        Com a matriz de covariância da \cref{eq:pca-covariancia}, encontramos e ordenamos os autovetores decrescentemente pelos seus autovalores na \cref{eq:pca-2}. Disso, isolamos os $k$ primeiros componentes dos autovetores.

        \begin{subequations}
            \label{eq:pca-2}
            \begin{equation}
                \label{eq:pca-autovalorvetor}
                \lambda, V = \texttt{sort}(\texttt{eig}(\Sigma))
            \end{equation}
            \begin{equation}
                \label{eq:pca-autovetor-k}
                V_k = V_{[:, 0:k]}
            \end{equation}
        \end{subequations}

        Utilizando os $k$ primeiros componentes, refazemos a matriz original na \cref{eq:pca-3} utilizando os $k$ primeiros autovetores e a matriz sem as médias, somando as médias removidas e finalmente produzindo a imagem final $X_f$.

        \begin{subequations}
            \label{eq:pca-3}
            \begin{equation}
                \label{eq:pca-saida}
                \hat{X} = V^T_k X_n
            \end{equation}
            \begin{equation}
                \label{eq:pca-recomposicao}
                X_f = (V_k \hat{X}) + \overline{X_j}
            \end{equation}
        \end{subequations}


    \subsection{Avaliando resultados}

        Agora, com as duas formas de se produzir uma imagem utilizando menos componentes, podemos avaliar a compressão. Para isso, salvamos cada imagem utilizando \emph{OpenCV}, calculamos a taxa de compressão e RMSE (\textit{Root mean square error}). Esses dados são impressos em um arquivo e podem ser comparados apenas dentro de uma mesma imagem (devida natureza de RMSE, dependendo muito da grandeza dos dados).

        Os testes foram automatizados através de um \textit{script} \texttt{bash}. Fazemos para cada imagem alguns valores de $k$: 32, 64, 128, 192 e 256, produzindo uma imagem de saída e um relatório diferente.


\section{Resultados}

    Os resultados obtidos serão analisados quanto a semelhança a original e quanto aos resultados numéricos no relatório (taxa de compressão, RMSE).
    Em todas as imagens, o resultado utilizando o SVD de \emph{NumPy} foi praticamente igual à nossa implementação, desviando muito pouco tanto em espaço ocupado, quanto RMSE.

    \subsection{Imagem \texttt{baboon}}

        Começando com a imagem \texttt{baboon}, para todo $k$ testado, a taxa de compressão foi menor que 1, ou seja, de fato a redução de componentes auxiliou a diminuição do tamanho do arquivo para esta imagem.

        Não há diferenças visuais notáveis para $k \geq 192$, mas conseguimos notar em $k = 128$ uma suavização (por exemplo, o brilho no nariz na \cref{fig:baboon-128}) e, para valores menores, uma perda de qualidade notável. No melhor caso ($k = 192$), a taxa de compressão foi $0.981$ com RMSE $7.2306$, o que representa uma diminuição de $30 kB$ -- considerável em um banco grande de imagens, desprezível para um usuário comum.

        \begin{figure}[h]
            \centering
            \begin{subfigure}{0.27\textwidth}
                \includegraphics[width=\textwidth,keepaspectratio]{baboon-128}
                \caption{$k=128$}
                \label{fig:baboon-128}
            \end{subfigure}
            \begin{subfigure}{0.27\textwidth}
                \includegraphics[width=\textwidth,keepaspectratio]{baboon}
                \caption{Original}
                \label{fig:baboon-original}
            \end{subfigure}
            \begin{subfigure}{0.27\textwidth}
                \includegraphics[width=\textwidth,keepaspectratio]{baboon-192}
                \caption{$k=192$}
                \label{fig:baboon-192}
            \end{subfigure}

            \caption{Comparativo entre $k=192$ e $k=128$ para \texttt{baboon}}
            \label{fig:baboon-comparativo}
        \end{figure}

    \subsection{Imagem \texttt{monalisa}}

        Com dimensões muito menores, para os mesmos valores de $k$, a imagem é bastante favorecida, já que o número de \textit{pixels} por componente é menor, o que intuitivamente implicaria numa representação melhor ou mais eficiente.

        No entanto, para $k \geq 128$, a taxa de compressão foi sempre maior ou igual 1 (igual em $k = 256$, que é a dimensão da imagem), o que é ineficiente. Para $k = 32$, há o efeito de suavização já mencionado e, para $k = 64$, não podemos notar artefatos, conseguindo taxa de compressão $0.965$ e RMSE $3.7023$, diminuindo apenas $5 kB$. É possível comparar com a imagem original nas \cref{fig:monalisa-64,fig:monalisa-original}.

    \subsection{Imagem \texttt{peppers}}

        A imagem possui dimensões iguais às do \texttt{baboon}, com resultados diferentes: não conseguimos notar diferenças visuais na imagem a partir de $k = 64$, oferecendo taxa de compressão $0.910$ e RMSE $6.8881$, economizando $20kB$. Para todo valor de $k$, temos um fator de compressão menor do que 1.

        Podemos tentar identificar diferenças para \texttt{peppers} na \cref{fig:peppers-comparativo}.

        \begin{figure}[H]
            \centering
            \begin{subfigure}{0.15\textwidth}
                \includegraphics[width=\textwidth,keepaspectratio]{peppers-64}
                \caption{$k=64$}
                \label{fig:peppers-64}
            \end{subfigure}
            \begin{subfigure}{0.15\textwidth}
                \includegraphics[width=\textwidth,keepaspectratio]{peppers}
                \caption{Original}
                \label{fig:peppers-original}
            \end{subfigure}
            \begin{subfigure}{0.15\textwidth}
                \includegraphics[width=\textwidth,keepaspectratio]{peppers-128}
                \caption{$k=128$}
                \label{fig:peppers-128}
            \end{subfigure}

            \caption{Comparativo entre $k=64$ e $k=128$ para \texttt{peppers}}
            \label{fig:peppers-comparativo}
        \end{figure}

    \subsection{Imagem \texttt{watch}}

        As dimensões dessa imagem são as maiores das imagens dadas, também é  produzida por computação gráfica. Para todo $k$, a taxa de compressão é maior que 1. Ou seja, mesmo com a maior quantidade de componentes, a redução de qualidade não auxilia na diminuição do espaço ocupado.

        \begin{figure}[h]
            \centering
            \begin{subfigure}{0.27\textwidth}
                \includegraphics[width=\textwidth,keepaspectratio]{watch-128}
                \caption{$k=128$}
                \label{fig:watch-128}
            \end{subfigure}
            \begin{subfigure}{0.27\textwidth}
                \includegraphics[width=\textwidth,keepaspectratio]{watch}
                \caption{original}
                \label{fig:watch-original}
            \end{subfigure}
            \begin{subfigure}{0.27\textwidth}
                \includegraphics[width=\textwidth,keepaspectratio]{watch-256}
                \caption{$k=256$}
                \label{fig:watch-256}
            \end{subfigure}

            \caption{Comparativo entre $k=256$ e $k=128$ para \texttt{watch}}
            \label{fig:comparativo-watch}
        \end{figure}

        É possível notar inclusive na imagem de $k=256$ (\cref{fig:watch-256}) componentes um ruído presente nas cores escuras. O melhor resultado foi para $k=32$, com taxa de compressão $1.296$ e RMSE $13.9669$, produzindo uma imagem cerca de $200kB$ mais pesada. No pior caso, a taxa de compressão foi $1.566$ para RMSE $2.8629$.

    \subsection{Imagem \texttt{mel}}

        Única imagem não fornecida pelo professor, capturada utilizando celular comum com grandes dimensões comparado às outras. Podemos notar que a taxa de compressão foi menor que 1 apenas para $k=32$ e $k=64$. Ao contrário de \texttt{watch}, a taxa de compressão foi mais estável, não passando de $1.2$.

        Não é possível notar mais diferenças em relação à original a partir de $k=192$, que possui taxa de compressão $1.114$ e RMSE $3.4381$, utilizando cerca de $200kB$ adicionais, como a \texttt{watch}.

        Podemos comparar os resultados mais próximos do ideal com a original na \cref{fig:mel-comparativo}. É possível notar ruído na parede na \cref{fig:mel-128}, mesmo em tamanho pequeno.

        \begin{figure}[H]
            \centering
            \begin{subfigure}{0.15\textwidth}
                \includegraphics[width=\textwidth,keepaspectratio]{mel-128}
                \caption{$k=128$}
                \label{fig:mel-128}
            \end{subfigure}
            \begin{subfigure}{0.15\textwidth}
                \includegraphics[width=\textwidth,keepaspectratio]{mel}
                \caption{Original}
                \label{fig:mel-original}
            \end{subfigure}
            \begin{subfigure}{0.15\textwidth}
                \includegraphics[width=\textwidth,keepaspectratio]{mel-192}
                \caption{$k=192$}
                \label{fig:mel-192}
            \end{subfigure}

            \caption{Comparativo entre $k=128$ e $k=192$ para \texttt{mel}}
            \label{fig:mel-comparativo}
        \end{figure}

    \subsection{Efeito em dimensões menores}

        Como ambas as imagens de dimensões grandes produziram taxas de compressão não satisfatórias, analisamos também versões menores das mesmas imagens (produzidas com o \emph{GIMP}, usando interpolação cúbica com escala menor, produzindo dimensões parecidas com a menor imagem: \texttt{monalisa}). Adicionamos o sufixo \texttt{-menor} nessas versões: \texttt{mel-menor} e \texttt{watch-menor}.

        Podemos notar uma melhoria na imagem \texttt{watch-menor} em relação a \texttt{watch}. Apesar de ainda possuir apenas taxas maiores ou iguais a 1 em todos os casos, o valor não passa de $1.3$ (antes o valor mais alto era $1.566$). Curiosamente, a pior taxa foi para $k=128$ com $1.275$ e a melhor, para $k=256$ (maior que a dimensão horizontal da imagem), com $1.0$, que possui RMSE de $0.0$, já que a imagem é integralmente reconstruída pelos seus componentes. A partir de $k=192$, não notamos diferenças em relação a original.

        Para a imagem \texttt{mel-menor}, apenas para $k=32$ temos uma taxa de compressão menor que 1, que é $0.998$ (com RMSE $7.5668$). Não conseguimos notar diferenças visuais para $k \geq 128$.

        Ou seja, nessas imagens não obtivemos uma melhora pela redução da dimensão: ainda há espaço extra ocupado em relação a imagem original.

    \subsection{Número de componentes}

        Como explicitado em aula, podemos comprovar a dificuldade para escolher um valor $k$. Há imagens em que visualmente não podemos notar artefatos com $k$ pequeno e outras que fazem muita diferença, mesmo em dimensões parecidas. Podemos perceber isso na comparação das imagens \texttt{monalisa} e \texttt{watch-menor} na \cref{fig:comparativo-watch-menor-monalisa}, cujas dimensões são parecidas e número de \textit{pixels} diferem em cerca de apenas 2 mil, mas notamos na \cref{fig:watch-menor-64} um ruído considerável.

        \begin{figure}[H]
            \centering
            \begin{subfigure}{0.23\textwidth}
                \includegraphics[width=\textwidth,keepaspectratio]{monalisa}
                \caption{\texttt{monalisa} original}
                \label{fig:monalisa-original}
            \end{subfigure}
            \begin{subfigure}{0.23\textwidth}
                \includegraphics[width=\textwidth,keepaspectratio]{monalisa-64}
                \caption{\texttt{monalisa} $k=64$}
                \label{fig:monalisa-64}
            \end{subfigure}

            \begin{subfigure}{0.23\textwidth}
                \includegraphics[width=\textwidth,keepaspectratio]{watch-menor}
                \caption{\texttt{watch-menor} original}
                \label{fig:watch-menor-original}
            \end{subfigure}
            \begin{subfigure}{0.23\textwidth}
                \includegraphics[width=\textwidth,keepaspectratio]{watch-menor-64}
                \caption{\texttt{watch-menor} $k=64$}
                \label{fig:watch-menor-64}
            \end{subfigure}

            \caption{Comparativo entre duas imagens de dimensões próximas}
            \label{fig:comparativo-watch-menor-monalisa}
        \end{figure}


\section{Conclusão}

    Conseguimos conluir que a análise de componentes principais (PCA) possibilita a compressão através da exclusão de componentes menos significativos da imagem.

    No entanto, é válido dizer que nem toda imagem é beneficiada pelo método, produzindo resultados que ocupam mais espaço enquanto reduzem a qualidade visual.

    O método é bastante eficiente para reduzir o espaço ocupado em um banco de muitas imagens de dimensões pequenas, pois a diminuição da qualidade é difícil de ser percebida nesses casos, mostrando-se muito bom para produzir \textit{thumbnails}, \textit{previews} de imagens, já que o algoritmo é bastante rápido nessas condições (em imagens grandes, o algoritmo é muito ineficiente).

\end{document}
